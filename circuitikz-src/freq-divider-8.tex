\begin{circuitikz}[american voltages]
    \ctikzset{resistors/scale=0.5}
    \ctikzset{multipoles/thickness=4}
    \ctikzset{multipoles/external pins thickness=2}
    \ctikzset{multipoles/dipchip/width=2}
    \draw (0,0) node[
        dipchip,
        num pins=18,
        hide numbers,
        external pins width=0,
    ](HC161){HC161};
    \node [right, font=\tiny] at (HC161.bpin 1) {A};
    \node [right, font=\tiny] at (HC161.bpin 2) {B};
    \node [right, font=\tiny] at (HC161.bpin 3) {C};
    \node [right, font=\tiny] at (HC161.bpin 4) {D};
    \node [right, font=\tiny] at (HC161.bpin 5) {ENP};
    \node [right, font=\tiny] at (HC161.bpin 6) {ENT};
    \node [right, font=\tiny] at (HC161.bpin 7) {$ \mathrm{\overline{LOAD}} $};
    \node [right, font=\tiny] at (HC161.bpin 8) {$ \mathrm{\overline{CLR}} $};
    \draw (HC161.bpin 9) -- ++(0, 0.1) -- ++(0.1,-0.1) coordinate(clklable) -- ++(-0.1,-0.1);
    \node [right, font=\tiny] at (clklable) {CLK};
    \node [left, font=\tiny] at (HC161.bpin 18) {QA};
    \node [left, font=\tiny] at (HC161.bpin 17) {QB};
    \node [left, font=\tiny] at (HC161.bpin 16) {QC};
    \node [left, font=\tiny] at (HC161.bpin 15) {QD};
    \node [left, font=\tiny] at (HC161.bpin 13) {RCO};

    \draw (HC161.pin 5) -- ++(-0.5,0) coordinate(ENP) -- ++(0,3) node[vcc]{VDD};
    \draw (HC161.pin 6) -- ++(-0.5,0) coordinate(ENT) to[short, -*] (ENP);
    \draw (HC161.pin 7) -- ++(-0.5,0) coordinate(NLOAD) to[short, -*] (ENT);
    \draw (HC161.pin 8) -- ++(-0.5,0) coordinate(CLR) to[short, -*] (NLOAD);
    \draw (HC161.pin 9) to[short, -o] ++(-1,0) node[left]{$ v_{in} $};
    \draw (HC161.pin 16) to[short, -o] ++(1,0) node[right]{$ v_{out} $};
\end{circuitikz}